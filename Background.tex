\section{Background and Related Work}
Our game is a form of Multi-Agent System (MAS), a class which emphasizes the joint action of several agents in an environment. To cooperate in MAS, AI agents can employ three types of message sharing: {\bf{1. Percepts}}, facts about the environment, such as the agent's location or intended action; {\bf{2. Events}}, a tuple representing an agent's state, action, and reward; {\bf{3. Learned policies}}, the learned utility of taking a particular action in a particular state from past experiences ~\cite{Tan93IndepVsCoop}. 

The work presented in this paper focuses on how the performance of a MAS can alter significantly with the introduction of limited message sharing. MAS can be seen in various scenarios, for example multi-agent foraging where several robots are supposed to discover particular rocks and bring them to a specific place~\cite{Panait05CooperativeMultiAgent}.


%Possibly room to extend for explination. Are we assuming the reader knows what joint action is? 

%This might be where we want to explain that we are playing Markov games where the environment is changing.

%Seems the paper should be organized: Intro -> Background (RL and MARL) -> Game Design -> Algorithms (Cooperative, Independent, Oracle)
